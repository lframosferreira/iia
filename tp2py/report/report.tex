\documentclass{article}
\usepackage{multicol}
\usepackage{subfig}

\usepackage{amsmath}
\usepackage{float}
\usepackage{graphicx}

\usepackage[colorlinks=true, allcolors=blue]{hyperref}

\title{Trabalho Prático II - Introdução à Inteligência Artificial}
\author{Luís Felipe Ramos Ferreira}
\date{\href{mailto:lframos.lf@gmail.com}{\texttt{lframos.lf@gmail.com}}
}

\begin{document}

\maketitle

\section{Introdução}

O Trabalho Prático II da disciplina de Introdução a Inteligência Artificial teve como objetivo a implementação do algoritmo de \texttt{Q-Learning}
para encontrar o melhor caminho entro um ponto inicial e um objetiov em uma mapa bidimensional.

\section{Implementação}

O projeto foi implementado na linguagem Python, versão 3.12.3. Um arquivo \texttt{requirements.txt} com os pacotes utilizados no ambiente virtual criado para desenvolvimento
está disponibilizado. O único pacote fora dos já disponibilizados por padrão na Linguagem foram Numpy, Pandas e Matplotlib. Instruções para rodar o programa estão disponibilizadas no
arquivo \texttt{README.md} disponibilizado.

\section{Q-learning}

bla

\begin{itemize}
	\item standard
	\item positive
	\item stochastic
\end{itemize}

\section{Estruturas e modelagem}


\section{Análise das diferentes políticas}

\section{Conclusão}

the last dance

\end{document}

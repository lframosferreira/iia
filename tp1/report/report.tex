\documentclass{article}

\usepackage{amsmath}
\usepackage{float}
\usepackage{graphicx}

\usepackage[colorlinks=true, allcolors=blue]{hyperref}

\title{Trabalho Prático I - Introdução à Inteligência Artificial}
\author{Luís Felipe Ramos Ferreira}
\date{\href{mailto:lframos.lf@gmail.com}{\texttt{lframos.lf@gmail.com}}
}

\begin{document}

\maketitle

\section{Introdução}

O Trabalho Prático I da disciplina de Introdução a Inteligência Artificial teve como objetivo a implementação de
5 algoritmos de busca diferentes em um problema de menor caminho entre dois pontos em um mapa bidimensional.

\section{Implementação}

O projeto foi implementado utilizando C++ e Python. O código principal, que contêm a implementação dos 5 algoritmos
citados foi escrito em C++, versão 17, e está todo contido no arquivo \texttt{main.cpp}. Arquivos utilitários utilizados para
realização de \textit{benchmarks} comparativos entre os algoritmos, criação dos gráficos para análise ds dados foram criados
utilizando Python, versão 3.12.3, e os pacotes utilizados foram manejados com o gerenciador de pacotes pip.

Instruções de como executar o programa principal e os arquivos utilitários, assim como a listagem de dependências, estão
presentes no arquivo \texttt{README.md}, no diretório raiz do projeto.

Os testes apresentados na análise de resultados foram feitos em uma máquina com Ubuntu 24.04.01 e 16GB de memória RAM, na CPU
11th Gen Intel i5-11400F (12) @ 4.400GHz.

\section{Descrição dos algoritmos}

Nesta seção, é feita uma breve descrição dos algoritmos utilizados e suas principais diferenças.
FAZER ISSO AQUI PRA ESTUDAR

\begin{itemize}
	\item \textit{Breadth First Search (BFS)}
	\item \textit{Iterative Depth Search (IDS)}
	\item \textit{Uniform COst Search (UCS)}
	\item \textit{Greedy}
	\item \(\text{A}^*\)
\end{itemize}


\section{Heurísticas utilizadas}

Nesta seção são apresentadas as heurísticas utilizadas nos algoritmos \textit{Greedy} e \(\text{A}^*\).

\begin{itemize}
	\item Distância \textit{Manhattan}
	\item algm outras
\end{itemize}

\section{Resultados}

\subsection{Número de estados expandidos}

\subsection{Tempo de execução}

\section{Discussão dos resultados}

blabla

\end{document}

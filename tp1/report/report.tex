\documentclass{article}

\usepackage{amsmath}
\usepackage{float}
\usepackage{graphicx}

\usepackage[colorlinks=true, allcolors=blue]{hyperref}

\title{Trabalho Prático I - Introdução à Inteligência Artificial}
\author{Luís Felipe Ramos Ferreira}
\date{\href{mailto:lframos.lf@gmail.com}{\texttt{lframos.lf@gmail.com}}
}

\begin{document}

\maketitle

\section{Introdução}

O Trabalho Prático I da disciplina de Introdução a Inteligência Artificial teve como objetivo a implementação de
5 algoritmos de busca diferentes em um problema de menor caminho entre dois pontos em um mapa bidimensional.

\section{Implementação}

O projeto foi implementado utilizando C++ e Python. O código principal, que contêm a implementação dos 5 algoritmos
citados foi escrito em C++, versão 17, e está todo contido no arquivo \texttt{main.cpp}. Arquivos utilitários utilizados para
realização de \textit{benchmarks} comparativos entre os algoritmos, criação dos gráficos para análise ds dados foram criados
utilizando Python, versão 3.12.3, e os pacotes utilizados foram manejados com o gerenciador de pacotes pip.

Instruções de como executar o programa principal e os arquivos utilitários, assim como a listagem de dependências, estão
presentes no arquivo \texttt{README.md}, no diretório raiz do projeto.

Os testes apresentados na análise de resultados foram feitos em uma máquina com Ubuntu 24.04.01 e 16GB de memória RAM, na CPU
11th Gen Intel i5-11400F (12) @ 4.400GHz.

\section{Descrição dos algoritmos}

Nesta seção, é feita uma breve descrição dos algoritmos utilizados e suas principais diferenças.
Para todos eles, consideramos um \textit{branch factor} \(b\) e
que a solução está no nível \(d\).

\begin{itemize}
	\item \textit{Breadth First Search (BFS)}
		A busca em largura é um algoritmo em que, a cada iteração,
		se expande o nó mais raso ainda não expandido na busca. Em termos informais,
		primeiro se expande a raiz, depois os sucessores da raiz, depois os sucessores
		dos sucessores da raiz, e assim se segue. O algoritmo pode ser implementado
		com uma fila, pois a ideia dele segue um arquitetura de FIFO (\textit{First In First Out}).
		No algoritmo podemos fazer algo chamado \textit{Early Goal Test}, em que checamos se um nó adicionado na fila
		já é o nó final antes de processá-lo ao sair da fila. Faz-se isso pois com isso
		o algoritmo não irá precisar adicionar mais nós à fila e nem processar os que já estão lá se o nó final
		já estiver pronto para ser processado.

		\begin{itemize}
			\item \texttt{Completo}
			\item \texttt{Ótimo}, se e somente se o custo seja uma função
				não decrescente da profundidade do nó (Por exemplo, na busca de menor caminho em
				um grafo sem pesos nas arestas.).
			\item \texttt{Complexidade}: tempo (\(\mathcal{O}(b^d)\)) e espaço (\(\mathcal{O}(b^d)\))
		\end{itemize}

	\item \textit{Iterative Depth Search (IDS)}
		A busca com profundidade iterativa é uma junção dos algoritmos
		BFS e DFS. Nele, aplicamos uma busca em profundidade iterativa, isto é, a cada iteração,
		aumentamos a profundidade permitida na busca. Nesse sentido,
		os benefícios da BFS e da DFS são combinado. 

		\begin{itemize}
			\item \texttt{Completo}.
		\item \texttt{Ótimo}, considerando custo crescente como no caso da busca em largura.
		\item \texttt{Complexidade}: tempo \(\mathcal{O}(b^d)\) e espaço
		\end{itemize}

	\item \textit{Uniform Cost Search (UCS)}

		O algoritmo de busca de custo uniforme é muito semelhante à BFS, mas
		o próximo nó expandido é na verdade o nó com menor custo até o momento. Para fazer isso,
		é necessário manter uma ordenação dos custos dos nós, e pra isso
		podemos utilizar uma fila de prioridades ou um \textit{heap} na
		implementação.

		\begin{itemize}
			\item \texttt{Completo}, se e somente se cada passo do algoritmo tem custo positivo (
				Considerando um grafo com pesos nas arestas, se houver uma aresta com peso negativo, o algoritmo
				não funcionaria).
			\item \texttt{Ótimo}, uma vez que seguimos o menor custo
			\item \texttt{Complexidade}: se \(C^*\) for a solução ótima,
				o pior caso de tempo e espaço é \(\mathcal{O}(b^{1 + \frac{C^*}{\epsilon}})\).
		\end{itemize}

	\item \textit{Greedy}

	\item \(\text{A}^*\)


\end{itemize}


\section{Heurísticas utilizadas}

Nesta seção são apresentadas as heurísticas utilizadas nos algoritmos \textit{Greedy} e \(\text{A}^*\).

\begin{itemize}
	\item Distância \textit{Manhattan}
	\item algm outras
\end{itemize}

\section{Resultados}

\subsection{Número de estados expandidos}

\subsection{Tempo de execução}

\section{Discussão dos resultados}

blabla

\end{document}
